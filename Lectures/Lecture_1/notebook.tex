
% Default to the notebook output style

    


% Inherit from the specified cell style.




    
\documentclass[11pt]{article}

    
    
    \usepackage[T1]{fontenc}
    % Nicer default font (+ math font) than Computer Modern for most use cases
    \usepackage{mathpazo}

    % Basic figure setup, for now with no caption control since it's done
    % automatically by Pandoc (which extracts ![](path) syntax from Markdown).
    \usepackage{graphicx}
    % We will generate all images so they have a width \maxwidth. This means
    % that they will get their normal width if they fit onto the page, but
    % are scaled down if they would overflow the margins.
    \makeatletter
    \def\maxwidth{\ifdim\Gin@nat@width>\linewidth\linewidth
    \else\Gin@nat@width\fi}
    \makeatother
    \let\Oldincludegraphics\includegraphics
    % Set max figure width to be 80% of text width, for now hardcoded.
    \renewcommand{\includegraphics}[1]{\Oldincludegraphics[width=.8\maxwidth]{#1}}
    % Ensure that by default, figures have no caption (until we provide a
    % proper Figure object with a Caption API and a way to capture that
    % in the conversion process - todo).
    \usepackage{caption}
    \DeclareCaptionLabelFormat{nolabel}{}
    \captionsetup{labelformat=nolabel}

    \usepackage{adjustbox} % Used to constrain images to a maximum size 
    \usepackage{xcolor} % Allow colors to be defined
    \usepackage{enumerate} % Needed for markdown enumerations to work
    \usepackage{geometry} % Used to adjust the document margins
    \usepackage{amsmath} % Equations
    \usepackage{amssymb} % Equations
    \usepackage{textcomp} % defines textquotesingle
    % Hack from http://tex.stackexchange.com/a/47451/13684:
    \AtBeginDocument{%
        \def\PYZsq{\textquotesingle}% Upright quotes in Pygmentized code
    }
    \usepackage{upquote} % Upright quotes for verbatim code
    \usepackage{eurosym} % defines \euro
    \usepackage[mathletters]{ucs} % Extended unicode (utf-8) support
    \usepackage[utf8x]{inputenc} % Allow utf-8 characters in the tex document
    \usepackage{fancyvrb} % verbatim replacement that allows latex
    \usepackage{grffile} % extends the file name processing of package graphics 
                         % to support a larger range 
    % The hyperref package gives us a pdf with properly built
    % internal navigation ('pdf bookmarks' for the table of contents,
    % internal cross-reference links, web links for URLs, etc.)
    \usepackage{hyperref}
    \usepackage{longtable} % longtable support required by pandoc >1.10
    \usepackage{booktabs}  % table support for pandoc > 1.12.2
    \usepackage[inline]{enumitem} % IRkernel/repr support (it uses the enumerate* environment)
    \usepackage[normalem]{ulem} % ulem is needed to support strikethroughs (\sout)
                                % normalem makes italics be italics, not underlines
    

    
    
    % Colors for the hyperref package
    \definecolor{urlcolor}{rgb}{0,.145,.698}
    \definecolor{linkcolor}{rgb}{.71,0.21,0.01}
    \definecolor{citecolor}{rgb}{.12,.54,.11}

    % ANSI colors
    \definecolor{ansi-black}{HTML}{3E424D}
    \definecolor{ansi-black-intense}{HTML}{282C36}
    \definecolor{ansi-red}{HTML}{E75C58}
    \definecolor{ansi-red-intense}{HTML}{B22B31}
    \definecolor{ansi-green}{HTML}{00A250}
    \definecolor{ansi-green-intense}{HTML}{007427}
    \definecolor{ansi-yellow}{HTML}{DDB62B}
    \definecolor{ansi-yellow-intense}{HTML}{B27D12}
    \definecolor{ansi-blue}{HTML}{208FFB}
    \definecolor{ansi-blue-intense}{HTML}{0065CA}
    \definecolor{ansi-magenta}{HTML}{D160C4}
    \definecolor{ansi-magenta-intense}{HTML}{A03196}
    \definecolor{ansi-cyan}{HTML}{60C6C8}
    \definecolor{ansi-cyan-intense}{HTML}{258F8F}
    \definecolor{ansi-white}{HTML}{C5C1B4}
    \definecolor{ansi-white-intense}{HTML}{A1A6B2}

    % commands and environments needed by pandoc snippets
    % extracted from the output of `pandoc -s`
    \providecommand{\tightlist}{%
      \setlength{\itemsep}{0pt}\setlength{\parskip}{0pt}}
    \DefineVerbatimEnvironment{Highlighting}{Verbatim}{commandchars=\\\{\}}
    % Add ',fontsize=\small' for more characters per line
    \newenvironment{Shaded}{}{}
    \newcommand{\KeywordTok}[1]{\textcolor[rgb]{0.00,0.44,0.13}{\textbf{{#1}}}}
    \newcommand{\DataTypeTok}[1]{\textcolor[rgb]{0.56,0.13,0.00}{{#1}}}
    \newcommand{\DecValTok}[1]{\textcolor[rgb]{0.25,0.63,0.44}{{#1}}}
    \newcommand{\BaseNTok}[1]{\textcolor[rgb]{0.25,0.63,0.44}{{#1}}}
    \newcommand{\FloatTok}[1]{\textcolor[rgb]{0.25,0.63,0.44}{{#1}}}
    \newcommand{\CharTok}[1]{\textcolor[rgb]{0.25,0.44,0.63}{{#1}}}
    \newcommand{\StringTok}[1]{\textcolor[rgb]{0.25,0.44,0.63}{{#1}}}
    \newcommand{\CommentTok}[1]{\textcolor[rgb]{0.38,0.63,0.69}{\textit{{#1}}}}
    \newcommand{\OtherTok}[1]{\textcolor[rgb]{0.00,0.44,0.13}{{#1}}}
    \newcommand{\AlertTok}[1]{\textcolor[rgb]{1.00,0.00,0.00}{\textbf{{#1}}}}
    \newcommand{\FunctionTok}[1]{\textcolor[rgb]{0.02,0.16,0.49}{{#1}}}
    \newcommand{\RegionMarkerTok}[1]{{#1}}
    \newcommand{\ErrorTok}[1]{\textcolor[rgb]{1.00,0.00,0.00}{\textbf{{#1}}}}
    \newcommand{\NormalTok}[1]{{#1}}
    
    % Additional commands for more recent versions of Pandoc
    \newcommand{\ConstantTok}[1]{\textcolor[rgb]{0.53,0.00,0.00}{{#1}}}
    \newcommand{\SpecialCharTok}[1]{\textcolor[rgb]{0.25,0.44,0.63}{{#1}}}
    \newcommand{\VerbatimStringTok}[1]{\textcolor[rgb]{0.25,0.44,0.63}{{#1}}}
    \newcommand{\SpecialStringTok}[1]{\textcolor[rgb]{0.73,0.40,0.53}{{#1}}}
    \newcommand{\ImportTok}[1]{{#1}}
    \newcommand{\DocumentationTok}[1]{\textcolor[rgb]{0.73,0.13,0.13}{\textit{{#1}}}}
    \newcommand{\AnnotationTok}[1]{\textcolor[rgb]{0.38,0.63,0.69}{\textbf{\textit{{#1}}}}}
    \newcommand{\CommentVarTok}[1]{\textcolor[rgb]{0.38,0.63,0.69}{\textbf{\textit{{#1}}}}}
    \newcommand{\VariableTok}[1]{\textcolor[rgb]{0.10,0.09,0.49}{{#1}}}
    \newcommand{\ControlFlowTok}[1]{\textcolor[rgb]{0.00,0.44,0.13}{\textbf{{#1}}}}
    \newcommand{\OperatorTok}[1]{\textcolor[rgb]{0.40,0.40,0.40}{{#1}}}
    \newcommand{\BuiltInTok}[1]{{#1}}
    \newcommand{\ExtensionTok}[1]{{#1}}
    \newcommand{\PreprocessorTok}[1]{\textcolor[rgb]{0.74,0.48,0.00}{{#1}}}
    \newcommand{\AttributeTok}[1]{\textcolor[rgb]{0.49,0.56,0.16}{{#1}}}
    \newcommand{\InformationTok}[1]{\textcolor[rgb]{0.38,0.63,0.69}{\textbf{\textit{{#1}}}}}
    \newcommand{\WarningTok}[1]{\textcolor[rgb]{0.38,0.63,0.69}{\textbf{\textit{{#1}}}}}
    
    
    % Define a nice break command that doesn't care if a line doesn't already
    % exist.
    \def\br{\hspace*{\fill} \\* }
    % Math Jax compatability definitions
    \def\gt{>}
    \def\lt{<}
    % Document parameters
    \title{w4111-L1-f2018-Introduction}
    
    
    

    % Pygments definitions
    
\makeatletter
\def\PY@reset{\let\PY@it=\relax \let\PY@bf=\relax%
    \let\PY@ul=\relax \let\PY@tc=\relax%
    \let\PY@bc=\relax \let\PY@ff=\relax}
\def\PY@tok#1{\csname PY@tok@#1\endcsname}
\def\PY@toks#1+{\ifx\relax#1\empty\else%
    \PY@tok{#1}\expandafter\PY@toks\fi}
\def\PY@do#1{\PY@bc{\PY@tc{\PY@ul{%
    \PY@it{\PY@bf{\PY@ff{#1}}}}}}}
\def\PY#1#2{\PY@reset\PY@toks#1+\relax+\PY@do{#2}}

\expandafter\def\csname PY@tok@w\endcsname{\def\PY@tc##1{\textcolor[rgb]{0.73,0.73,0.73}{##1}}}
\expandafter\def\csname PY@tok@c\endcsname{\let\PY@it=\textit\def\PY@tc##1{\textcolor[rgb]{0.25,0.50,0.50}{##1}}}
\expandafter\def\csname PY@tok@cp\endcsname{\def\PY@tc##1{\textcolor[rgb]{0.74,0.48,0.00}{##1}}}
\expandafter\def\csname PY@tok@k\endcsname{\let\PY@bf=\textbf\def\PY@tc##1{\textcolor[rgb]{0.00,0.50,0.00}{##1}}}
\expandafter\def\csname PY@tok@kp\endcsname{\def\PY@tc##1{\textcolor[rgb]{0.00,0.50,0.00}{##1}}}
\expandafter\def\csname PY@tok@kt\endcsname{\def\PY@tc##1{\textcolor[rgb]{0.69,0.00,0.25}{##1}}}
\expandafter\def\csname PY@tok@o\endcsname{\def\PY@tc##1{\textcolor[rgb]{0.40,0.40,0.40}{##1}}}
\expandafter\def\csname PY@tok@ow\endcsname{\let\PY@bf=\textbf\def\PY@tc##1{\textcolor[rgb]{0.67,0.13,1.00}{##1}}}
\expandafter\def\csname PY@tok@nb\endcsname{\def\PY@tc##1{\textcolor[rgb]{0.00,0.50,0.00}{##1}}}
\expandafter\def\csname PY@tok@nf\endcsname{\def\PY@tc##1{\textcolor[rgb]{0.00,0.00,1.00}{##1}}}
\expandafter\def\csname PY@tok@nc\endcsname{\let\PY@bf=\textbf\def\PY@tc##1{\textcolor[rgb]{0.00,0.00,1.00}{##1}}}
\expandafter\def\csname PY@tok@nn\endcsname{\let\PY@bf=\textbf\def\PY@tc##1{\textcolor[rgb]{0.00,0.00,1.00}{##1}}}
\expandafter\def\csname PY@tok@ne\endcsname{\let\PY@bf=\textbf\def\PY@tc##1{\textcolor[rgb]{0.82,0.25,0.23}{##1}}}
\expandafter\def\csname PY@tok@nv\endcsname{\def\PY@tc##1{\textcolor[rgb]{0.10,0.09,0.49}{##1}}}
\expandafter\def\csname PY@tok@no\endcsname{\def\PY@tc##1{\textcolor[rgb]{0.53,0.00,0.00}{##1}}}
\expandafter\def\csname PY@tok@nl\endcsname{\def\PY@tc##1{\textcolor[rgb]{0.63,0.63,0.00}{##1}}}
\expandafter\def\csname PY@tok@ni\endcsname{\let\PY@bf=\textbf\def\PY@tc##1{\textcolor[rgb]{0.60,0.60,0.60}{##1}}}
\expandafter\def\csname PY@tok@na\endcsname{\def\PY@tc##1{\textcolor[rgb]{0.49,0.56,0.16}{##1}}}
\expandafter\def\csname PY@tok@nt\endcsname{\let\PY@bf=\textbf\def\PY@tc##1{\textcolor[rgb]{0.00,0.50,0.00}{##1}}}
\expandafter\def\csname PY@tok@nd\endcsname{\def\PY@tc##1{\textcolor[rgb]{0.67,0.13,1.00}{##1}}}
\expandafter\def\csname PY@tok@s\endcsname{\def\PY@tc##1{\textcolor[rgb]{0.73,0.13,0.13}{##1}}}
\expandafter\def\csname PY@tok@sd\endcsname{\let\PY@it=\textit\def\PY@tc##1{\textcolor[rgb]{0.73,0.13,0.13}{##1}}}
\expandafter\def\csname PY@tok@si\endcsname{\let\PY@bf=\textbf\def\PY@tc##1{\textcolor[rgb]{0.73,0.40,0.53}{##1}}}
\expandafter\def\csname PY@tok@se\endcsname{\let\PY@bf=\textbf\def\PY@tc##1{\textcolor[rgb]{0.73,0.40,0.13}{##1}}}
\expandafter\def\csname PY@tok@sr\endcsname{\def\PY@tc##1{\textcolor[rgb]{0.73,0.40,0.53}{##1}}}
\expandafter\def\csname PY@tok@ss\endcsname{\def\PY@tc##1{\textcolor[rgb]{0.10,0.09,0.49}{##1}}}
\expandafter\def\csname PY@tok@sx\endcsname{\def\PY@tc##1{\textcolor[rgb]{0.00,0.50,0.00}{##1}}}
\expandafter\def\csname PY@tok@m\endcsname{\def\PY@tc##1{\textcolor[rgb]{0.40,0.40,0.40}{##1}}}
\expandafter\def\csname PY@tok@gh\endcsname{\let\PY@bf=\textbf\def\PY@tc##1{\textcolor[rgb]{0.00,0.00,0.50}{##1}}}
\expandafter\def\csname PY@tok@gu\endcsname{\let\PY@bf=\textbf\def\PY@tc##1{\textcolor[rgb]{0.50,0.00,0.50}{##1}}}
\expandafter\def\csname PY@tok@gd\endcsname{\def\PY@tc##1{\textcolor[rgb]{0.63,0.00,0.00}{##1}}}
\expandafter\def\csname PY@tok@gi\endcsname{\def\PY@tc##1{\textcolor[rgb]{0.00,0.63,0.00}{##1}}}
\expandafter\def\csname PY@tok@gr\endcsname{\def\PY@tc##1{\textcolor[rgb]{1.00,0.00,0.00}{##1}}}
\expandafter\def\csname PY@tok@ge\endcsname{\let\PY@it=\textit}
\expandafter\def\csname PY@tok@gs\endcsname{\let\PY@bf=\textbf}
\expandafter\def\csname PY@tok@gp\endcsname{\let\PY@bf=\textbf\def\PY@tc##1{\textcolor[rgb]{0.00,0.00,0.50}{##1}}}
\expandafter\def\csname PY@tok@go\endcsname{\def\PY@tc##1{\textcolor[rgb]{0.53,0.53,0.53}{##1}}}
\expandafter\def\csname PY@tok@gt\endcsname{\def\PY@tc##1{\textcolor[rgb]{0.00,0.27,0.87}{##1}}}
\expandafter\def\csname PY@tok@err\endcsname{\def\PY@bc##1{\setlength{\fboxsep}{0pt}\fcolorbox[rgb]{1.00,0.00,0.00}{1,1,1}{\strut ##1}}}
\expandafter\def\csname PY@tok@kc\endcsname{\let\PY@bf=\textbf\def\PY@tc##1{\textcolor[rgb]{0.00,0.50,0.00}{##1}}}
\expandafter\def\csname PY@tok@kd\endcsname{\let\PY@bf=\textbf\def\PY@tc##1{\textcolor[rgb]{0.00,0.50,0.00}{##1}}}
\expandafter\def\csname PY@tok@kn\endcsname{\let\PY@bf=\textbf\def\PY@tc##1{\textcolor[rgb]{0.00,0.50,0.00}{##1}}}
\expandafter\def\csname PY@tok@kr\endcsname{\let\PY@bf=\textbf\def\PY@tc##1{\textcolor[rgb]{0.00,0.50,0.00}{##1}}}
\expandafter\def\csname PY@tok@bp\endcsname{\def\PY@tc##1{\textcolor[rgb]{0.00,0.50,0.00}{##1}}}
\expandafter\def\csname PY@tok@fm\endcsname{\def\PY@tc##1{\textcolor[rgb]{0.00,0.00,1.00}{##1}}}
\expandafter\def\csname PY@tok@vc\endcsname{\def\PY@tc##1{\textcolor[rgb]{0.10,0.09,0.49}{##1}}}
\expandafter\def\csname PY@tok@vg\endcsname{\def\PY@tc##1{\textcolor[rgb]{0.10,0.09,0.49}{##1}}}
\expandafter\def\csname PY@tok@vi\endcsname{\def\PY@tc##1{\textcolor[rgb]{0.10,0.09,0.49}{##1}}}
\expandafter\def\csname PY@tok@vm\endcsname{\def\PY@tc##1{\textcolor[rgb]{0.10,0.09,0.49}{##1}}}
\expandafter\def\csname PY@tok@sa\endcsname{\def\PY@tc##1{\textcolor[rgb]{0.73,0.13,0.13}{##1}}}
\expandafter\def\csname PY@tok@sb\endcsname{\def\PY@tc##1{\textcolor[rgb]{0.73,0.13,0.13}{##1}}}
\expandafter\def\csname PY@tok@sc\endcsname{\def\PY@tc##1{\textcolor[rgb]{0.73,0.13,0.13}{##1}}}
\expandafter\def\csname PY@tok@dl\endcsname{\def\PY@tc##1{\textcolor[rgb]{0.73,0.13,0.13}{##1}}}
\expandafter\def\csname PY@tok@s2\endcsname{\def\PY@tc##1{\textcolor[rgb]{0.73,0.13,0.13}{##1}}}
\expandafter\def\csname PY@tok@sh\endcsname{\def\PY@tc##1{\textcolor[rgb]{0.73,0.13,0.13}{##1}}}
\expandafter\def\csname PY@tok@s1\endcsname{\def\PY@tc##1{\textcolor[rgb]{0.73,0.13,0.13}{##1}}}
\expandafter\def\csname PY@tok@mb\endcsname{\def\PY@tc##1{\textcolor[rgb]{0.40,0.40,0.40}{##1}}}
\expandafter\def\csname PY@tok@mf\endcsname{\def\PY@tc##1{\textcolor[rgb]{0.40,0.40,0.40}{##1}}}
\expandafter\def\csname PY@tok@mh\endcsname{\def\PY@tc##1{\textcolor[rgb]{0.40,0.40,0.40}{##1}}}
\expandafter\def\csname PY@tok@mi\endcsname{\def\PY@tc##1{\textcolor[rgb]{0.40,0.40,0.40}{##1}}}
\expandafter\def\csname PY@tok@il\endcsname{\def\PY@tc##1{\textcolor[rgb]{0.40,0.40,0.40}{##1}}}
\expandafter\def\csname PY@tok@mo\endcsname{\def\PY@tc##1{\textcolor[rgb]{0.40,0.40,0.40}{##1}}}
\expandafter\def\csname PY@tok@ch\endcsname{\let\PY@it=\textit\def\PY@tc##1{\textcolor[rgb]{0.25,0.50,0.50}{##1}}}
\expandafter\def\csname PY@tok@cm\endcsname{\let\PY@it=\textit\def\PY@tc##1{\textcolor[rgb]{0.25,0.50,0.50}{##1}}}
\expandafter\def\csname PY@tok@cpf\endcsname{\let\PY@it=\textit\def\PY@tc##1{\textcolor[rgb]{0.25,0.50,0.50}{##1}}}
\expandafter\def\csname PY@tok@c1\endcsname{\let\PY@it=\textit\def\PY@tc##1{\textcolor[rgb]{0.25,0.50,0.50}{##1}}}
\expandafter\def\csname PY@tok@cs\endcsname{\let\PY@it=\textit\def\PY@tc##1{\textcolor[rgb]{0.25,0.50,0.50}{##1}}}

\def\PYZbs{\char`\\}
\def\PYZus{\char`\_}
\def\PYZob{\char`\{}
\def\PYZcb{\char`\}}
\def\PYZca{\char`\^}
\def\PYZam{\char`\&}
\def\PYZlt{\char`\<}
\def\PYZgt{\char`\>}
\def\PYZsh{\char`\#}
\def\PYZpc{\char`\%}
\def\PYZdl{\char`\$}
\def\PYZhy{\char`\-}
\def\PYZsq{\char`\'}
\def\PYZdq{\char`\"}
\def\PYZti{\char`\~}
% for compatibility with earlier versions
\def\PYZat{@}
\def\PYZlb{[}
\def\PYZrb{]}
\makeatother


    % Exact colors from NB
    \definecolor{incolor}{rgb}{0.0, 0.0, 0.5}
    \definecolor{outcolor}{rgb}{0.545, 0.0, 0.0}



    
    % Prevent overflowing lines due to hard-to-break entities
    \sloppy 
    % Setup hyperref package
    \hypersetup{
      breaklinks=true,  % so long urls are correctly broken across lines
      colorlinks=true,
      urlcolor=urlcolor,
      linkcolor=linkcolor,
      citecolor=citecolor,
      }
    % Slightly bigger margins than the latex defaults
    
    \geometry{verbose,tmargin=1in,bmargin=1in,lmargin=1in,rmargin=1in}
    
    

    \begin{document}
    
    
    \maketitle
    
    

    
    \section{\texorpdfstring{Introduction to Databases \(-\) Lecture 1:
Introduction}{Introduction to Databases - Lecture 1: Introduction}}\label{introduction-to-databases---lecture-1-introduction}

    \subsection{Lecture Overview}\label{lecture-overview}

\begin{enumerate}
\def\labelenumi{\arabic{enumi}.}
\tightlist
\item
  Course, sections and waitlist.
\item
  The course: course objectives, logistics, assignments/grading,
  material and outline.
\item
  Overview: data, data models, databases and database management
  systems.
\item
  Some sample code and database connections.
\item
  Preview of four applications we will "develop" to learn concepts.
\item
  Homework 1 setup.
\end{enumerate}

    \subsection{Course, Section and
Waitlist}\label{course-section-and-waitlist}

\begin{itemize}
\item
  There are 3 sections for this course:

  \begin{itemize}
  \tightlist
  \item
    COMS 4111 W sec:003 \(-\) The "standard" section with in-person
    lectures.
  \item
    COMS 4111 W sec:H03 \(-\) A hybrid section. This means that the
    lectures are recorded and you do not need to attend in-person.
  \item
    COMS 4111 W sec:V03 \(-\) \href{https://cvn.columbia.edu/}{Columbia
    Video Network} version of the course.
  \end{itemize}
\item
  All sections are effectively the same course/section.

  \begin{itemize}
  \tightlist
  \item
    Exactly the same lectures, examples, course material, etc.
  \item
    All students can access the recorded lectures.
  \item
    Same exams and homework assignments.
  \item
    Same instructor and course assistants.
  \item
    Same office hours.
  \end{itemize}
\item
  The administrative staff of the Dept. of Computer Science
  \textbf{directly} manages enrollment in the sections and the
  waitlists.

  \begin{itemize}
  \tightlist
  \item
    COMS W4111 is ALWAYS oversubscribed and has huge waitlists.
  \item
    The course is a requirement for several majors and tracks.
  \item
    There are complex rules about prioritizing students for course
    enrollment, and also which students can take the hybrid section.
  \end{itemize}
\item
  The limiting factor on the standard, in-person lecture enrollment is
  the size of the room. We have to comply with university, city and
  state fire and safety codes.
\item
  Summary: Contact Jessica Rosa in the Dept. of Computer Science if you
  want to discuss the waitlist or enrollment in sections.
\end{itemize}

    \subsection{The Course}\label{the-course}

From the
\href{http://bulletin.columbia.edu/columbia-college/departments-instruction/computer-science/\#coursestext}{Columbia
University Bulletin}

"The fundamentals of database design and application development using
databases: entity-relationship modeling, logical design of relational
databases, relational data definition and manipulation languages, SQL,
query processing, transaction processing. Programming projects are
required."

    \subsubsection{About your Instructor}\label{about-your-instructor}

\begin{itemize}
\tightlist
\item
  Donald F. Ferguson

  \begin{itemize}
  \tightlist
  \item
    \href{https://www.ansys.com/}{Ansys, Inc.}: Senior Technical Fellow,
    Chief SW Architect.
  \item
    \emph{Adjunct Professor,} Dept. of Computer Science.
  \end{itemize}
\item
  Contact information:

  \begin{itemize}
  \tightlist
  \item
    Email: dff9@columbia.edu
  \item
    Slack: https://dff-columbia.slack.com/
  \end{itemize}
\item
  Academic experience

  \begin{itemize}
  \tightlist
  \item
    Ph.D. in Computer Science, Columbia University, 1989
  \item
    Joined Columbia as full time \emph{Professor of Professional
    Practice}, 01-Jan-2018
  \item
    8 semesters as an adjunct professor teaching

    \begin{itemize}
    \tightlist
    \item
      \emph{E6998: Topics in Computer Science}

      \begin{itemize}
      \tightlist
      \item
        Cloud Computing
      \item
        Web and Internet Application Development
      \item
        Web Application Servers and Applications
      \item
        Microservices
      \end{itemize}
    \item
      \emph{W4111 - Introduction to Databases}
    \end{itemize}
  \item
    \emph{Professor of Professional Practice in Computer Science,}
    January 2018

    \begin{itemize}
    \tightlist
    \item
      \emph{E1006 - Introduction to Computing for Engineers and Applied
      Scientists using Python}
    \item
      \emph{W4111 - Introduction to Databases}
    \item
      \_E6156 - Topics in SW Engineering: Microservices and Cloud
      Applications
    \end{itemize}
  \end{itemize}
\item
  30 years industry experience

  \begin{itemize}
  \tightlist
  \item
    \href{https://en.wikipedia.org/wiki/IBM_Fellow}{IBM Fellow}, Chief
    Architect for {[}IBM Software
    Group{]}(https://en.wikipedia.org/wiki/IBM\_Software\_Group\_(SWG)
  \item
    Microsoft Technical Fellow
  \item
    Executive Vice President, Chief Technology Officer,
    \href{https://www.ca.com/us.html}{CA Technologies}
  \item
    Vice President, CTO, Senior Fellow,
    \href{https://en.wikipedia.org/wiki/Dell_Software}{Dell Software
    Group}
  \item
    Co-Founder and CTO, \href{https://seekatv.com/}{Seeka TV}
  \item
    Senior Technical Fellow, Chief SW Architect,
    \href{http://www.ansys.com}{Ansys, Inc.}
  \end{itemize}
\item
  Publications

  \begin{itemize}
  \tightlist
  \item
    Approximately 60 technical publications.
  \item
    Authored, co-authored several standards in web applications and web
    services.
  \item
    \emph{Web Services Platform Architecture: SOAP, WSDL, WS-Policy,
    WS-Addressing, WS-BPEL, WS-Reliable Messaging, and More,} ISBN-13:
    978-0131488748, 2005, Pearson Education.
  \item
    12 patents.
  \end{itemize}
\item
  Personal and hobbies

  \begin{itemize}
  \tightlist
  \item
    Two amazing daughters (One is Barnard student. One is a junior in
    high school)
  \item
    Interested in languages. Speak Spanish reasonably well and trying to
    learn Arabic.
  \item
    Black Belt in Kenpo Karate
  \item
    Amateur astronomy
  \item
    Road bicycling
  \item
    Officer in the New York Guard
  \end{itemize}
\end{itemize}

    \begin{longtable}[]{@{}c@{}}
\toprule
\tabularnewline
\midrule
\endhead
\textbf{About Me}\tabularnewline
\bottomrule
\end{longtable}

    \subsubsection{About this Course}\label{about-this-course}

\begin{itemize}
\tightlist
\item
  This course is foundational, and will teach you the core concepts in

  \begin{itemize}
  \tightlist
  \item
    Data modeling
  \item
    Data model implementation; Data manipulation.
  \item
    Different database models and database management systems.
  \item
    Implementation of data centric applications and database management
    systems.
  \end{itemize}
\item
  ANY non-trivial application

  \begin{itemize}
  \tightlist
  \item
    Requires a well-designed data model.
  \item
    Implements a data model and manipulates data.
  \item
    Uses a database management system.
  \end{itemize}
\item
  Understanding databases and database management are core to the
  ``hottest fields'' in computer science, e.g.

  \begin{itemize}
  \tightlist
  \item
    Data science
  \item
    Machine learning
  \item
    Intelligent (Autonomous) systems
  \item
    Internet-of-Things
  \item
    Cybersecurity
  \item
    Cloud Computing
  \end{itemize}
\item
  University courses on databases sometimes focus on theory and abstract
  concepts. This course will cover theory, but the emphasis will be on:

  \begin{itemize}
  \tightlist
  \item
    Practical, hands-on applications of databases.
  \item
    Developing and understanding database centric applications.
  \item
    Patterns and best practices.
  \end{itemize}
\item
   \textbf{Personal perspective}

  \begin{itemize}
  \tightlist
  \item
    A large percent of my career has been spent figuring out or leading
    teams that figured out how to model, implement and manipulate data.
  \item
    I have used the information in this class more than anything else I
    have learned.
  \item
    This will likely be true for you. 
  \end{itemize}
\end{itemize}

    \subsubsection{Course Objectives}\label{course-objectives}

\begin{itemize}
\item
  Have fun, learn a lot and come to appreciate and enjoy some amazing
  technology.
\item
  Provide a foundation that allows you to succeed in future courses.
  This is an \emph{introduction} to databases. The technology is crucial
  for

  \begin{itemize}
  \tightlist
  \item
    Advanced database classes
  \item
    Machine learning
  \item
    Big data, data analysis
  \item
    Numerical, simulation and analytics in operations research,
    engineering, economics, finance, life sciences, etc.
  \end{itemize}
\item
  Enable you to successfully apply the technology in your work and
  profession.
\end{itemize}

Have cool stuff to talk about on interviews and resumes.

    \subsubsection{Organization and
Logistics}\label{organization-and-logistics}

\begin{itemize}
\item
  Lecture: Thursday, 7:00 PM to 9:00 PM, 207 Math
\item
  Instructor: Donald F. Ferguson (dff@cs.columbia.edu), 624 CEPSR
\item
  Office Hours:

  \begin{itemize}
  \tightlist
  \item
    Thursday, 5:00-7:00 PM
  \item
    By appointment, as needed, as available. I typically post on Piazza
    when I will have extra availability.
  \end{itemize}
\item
  Collaboration/Contact

  \begin{itemize}
  \tightlist
  \item
    The class is on
    \href{https://piazza.com/class/jpltphl4wz76cm}{Piazza}

    \begin{itemize}
    \tightlist
    \item
      General questions
    \item
      Clarification of homeworks, class material, etc.
    \end{itemize}
  \item
    Slack, for quick messages and questions.

    \begin{itemize}
    \tightlist
    \item
      Direct message to me, and
    \item
      Please join slack and the channel
      \href{https://join.slack.com/t/dff-columbia/shared_invite/enQtMjg0Mzk4MTQwMzQxLTZlNzk3OTZmNWE2NzNmNzViZmJlMWVmNWVlZmUxZTU5NjkwYjQ1YTdjMzA3ZTMzZDM3ZmIwYzAyYjIwYTNkZDI}{\#w4111s19}
      for quick questions/comments to class.
    \end{itemize}
  \item
    The course lectures, sample code, etc. will be on GitHub.

    \begin{itemize}
    \tightlist
    \item
      GitHub home page
      (https://donald-f-ferguson.github.io/w4111-Databases/)
    \item
      Repository/Project
      (https://github.com/donald-f-ferguson/w4111-Databases)
    \end{itemize}
  \item
    Course Assistants

    \begin{itemize}
    \tightlist
    \item
      TBA
    \item
      Will join Piazza and Slack channel, and announce office hours,
      contact info, etc.
    \end{itemize}
  \end{itemize}
\end{itemize}

    \subsubsection{Assignments, Exams and
Grades}\label{assignments-exams-and-grades}

\begin{itemize}
\tightlist
\item
  Point value of assignments and exams

  \begin{itemize}
  \tightlist
  \item
    50\%: Homework assignments

    \begin{itemize}
    \tightlist
    \item
      Approximately one HW every two weeks, for a total of 5 or 6.
    \item
      Mix of programming assignments and questions.
    \end{itemize}
  \item
    Exams: All exams are "take home exams."

    \begin{itemize}
    \tightlist
    \item
      Mix of short programming assignments and questions.
    \item
      20\% of grade is midterm exam score.
    \item
      30\% is final exam score.
    \end{itemize}
  \item
    Extra-credit: Class participation, contributions on Piazza and
    office hour participation earn extra-credit points.
  \end{itemize}
\item
  Late submission

  \begin{itemize}
  \tightlist
  \item
    You have a total of 5 grace days to apply for all homeworks.
  \item
    1 minute past the due date counts for 1 day. 24 hours + 1 minute
    counts for two days.
  \item
    You cannot use grace days for midterm, final exam or extra-credit
    assignments.
  \item
    NOTE:

    \begin{itemize}
    \tightlist
    \item
      Respect for the individual is paramount.
    \item
      We will always accommodate illness, family emergencies, etc.
    \end{itemize}
  \end{itemize}
\end{itemize}

    \subsubsection{Environment and material}\label{environment-and-material}

\begin{itemize}
\tightlist
\item
  Course material

  \begin{itemize}
  \tightlist
  \item
    Lectures and references: Textbooks are dated and also lack realistic
    scenarios and concepts.
  \item
    Textbook:

    \begin{itemize}
    \tightlist
    \item
      \emph{Database Management Systems, 3rd Edition}, Ramakrishnan and
      Gehrke, ISBN: 978-0072465631
    \item
      We will cover a subset of the material in the textbook, and in a
      different order.
    \item
      I will bring in examples from industry, engineering and practical
      experience.
    \end{itemize}
  \item
    Lecture material and examples:

    \begin{itemize}
    \tightlist
    \item
      Will be Jupyter Notebooks (http://jupyter.org/) for some lectures.
    \item
      Slides (PowerPoint) for other lectures.
    \item
      Notebooks, slides, sample code, etc. will be available on
      \href{https://github.com/donald-f-ferguson/w4111-Databases}{GitHub
      project} for the course. The easiest approach is to clone the
      project repository.
    \end{itemize}
  \end{itemize}
\item
  Project and development

  \begin{itemize}
  \tightlist
  \item
    I will primarily use Python. I strongly recommend you do the same.
  \item
    You can use the language of your choice, but my ability to help
    diminishes if you choose a language other than JavaScript, Python or
    Java.
  \end{itemize}
\item
  Development environment

  \begin{itemize}
  \tightlist
  \item
    Database engines and tools

    \begin{itemize}
    \tightlist
    \item
      We will start with the relational data model. Please install

      \begin{itemize}
      \tightlist
      \item
        \href{https://dev.mysql.com/doc/refman/5.7/en/installing.html}{MySQL}
        and
      \item
        \href{https://dev.mysql.com/doc/workbench/en/wb-installing.html}{MySQL
        Workbench}.
      \item
        You can also use Sequel Pro on in place of SQL Workbench (on
        Mac).
      \end{itemize}
    \item
      We will also use \href{https://neo4j.com/}{Neo4j} and Redis
      (https://redis.io/), but you do not need to install now. I will
      try to cover DynamoDB and Firebase Realtime Database, which are
      cloud DBs.
    \end{itemize}
  \item
    \href{https://en.wikipedia.org/wiki/Integrated_development_environment}{Integrated
    Development Environments}

    \begin{itemize}
    \tightlist
    \item
      The \href{https://www.jetbrains.com/student/}{JetBrains} tools are
      free for students, and useful for Python, JavaScript and Java.
      \textbf{I strongly recommend you use one of these tools for
      developing code.}
    \item
      \href{https://www.eclipse.org/}{Eclipse} is an alternative.
    \item
      I will use Ananconda/Jupyter for most of the lectures and
      examples. You can choose to install

      \begin{itemize}
      \tightlist
      \item
        Install Python 3
      \item
        \href{https://www.anaconda.com/distribution}{Anaconda Community
        Distribution}
      \item
        Ananconda includes the
        \href{https://github.com/spyder-ide/spyder}{Spyder} for Python,
        which is sufficient for the course.
      \end{itemize}
    \end{itemize}
  \end{itemize}
\item
  I will provide links to online material providing definitions and
  optional information. I often use Wikipedia because it is a good
  starting point and has many references.
\end{itemize}

    \subsubsection{Course Outline}\label{course-outline}

\textbf{Module I: Foundational Concepts}

\begin{enumerate}
\def\labelenumi{\arabic{enumi}.}
\item
  Introduction to databases, role in applications, type of DB
  applications and overall system software architecture.
\item
  Information and data modeling and best practices, focusing on
  supporting application scenarios.
\item
  Relational data model (theory), Relational Database Management
  Systems, Structured Query Language, data query and update scenarios.
\item
  Extended topics in SQL and RDBMS (performance, security, constraints,
  triggers, connection management, etc).
\end{enumerate}

\textbf{Module II: Database Management System
Implementation/Architecture}

\begin{enumerate}
\def\labelenumi{\arabic{enumi}.}
\setcounter{enumi}{4}
\item
  Storage management, disk management, buffer management, indexes.
\item
  Query processing and optimization: Query evaluation, query parsing and
  parse trees, operator implementation algorithms, query rewrite, query
  optimization techniques.
\item
  Concurrency control and transaction management.
\end{enumerate}

\textbf{Module III: NoSQL Database Overview}

\begin{enumerate}
\def\labelenumi{\arabic{enumi}.}
\setcounter{enumi}{7}
\item
  Overview, graph databases, Redis.
\item
  Amazon S3, Amazon DynamoDB, Google Firebase/Cloud Firestore.
\end{enumerate}

\textbf{Module IV: Decision Support, Data Analysis}

\begin{enumerate}
\def\labelenumi{\arabic{enumi}.}
\setcounter{enumi}{9}
\tightlist
\item
  Overview of schema denormalization, OLAP cubes, data analytics,
  machine learning.
\end{enumerate}

    \subsection{Data, Databases, Data Models and Database Management
Systems}\label{data-databases-data-models-and-database-management-systems}

\subsubsection{What is Data?}\label{what-is-data}

\begin{longtable}[]{@{}c@{}}
\toprule
\tabularnewline
\midrule
\endhead
\textbf{What is Data}\tabularnewline
\bottomrule
\end{longtable}

    \subsubsection{Examples of Data}\label{examples-of-data}

\begin{longtable}[]{@{}c@{}}
\toprule
\tabularnewline
\midrule
\endhead
\textbf{Examples}\tabularnewline
\bottomrule
\end{longtable}

    \subsubsection{Some Ways to Think about
Data}\label{some-ways-to-think-about-data}

\paragraph{Structures vs Unstructured}\label{structures-vs-unstructured}

\begin{longtable}[]{@{}c@{}}
\toprule
\tabularnewline
\midrule
\endhead
\textbf{Structured vs Unstructured}\tabularnewline
\bottomrule
\end{longtable}

We will focus on structured data and some of semi-structured. - Need to
constrain the scope for a one semester class. - Concepts are
foundational and apply to semi-structured and unstructured data. -
Unstructured data scenarios typically require extracting or extending
with structured data.

\paragraph{\texorpdfstring{\href{https://en.wikipedia.org/wiki/Metadata}{Data
versus Metadata}}{Data versus Metadata}}\label{data-versus-metadata}

Metadata~is "data~information that provides information about other
data". Three distinct types of metadata exist:~descriptive
metadata,~structural metadata, and~administrative metadata. -
Descriptive metadata describes a resource for purposes such as discovery
and identification. It can include elements such as title, abstract,
author, and keywords. - Structural metadata is metadata about containers
of data and indicates how compound objects are put together, for
example, how pages are ordered to form chapters. It describes the types,
versions, relationships and other characteristics of digital materials.
- Administrative metadata provides information to help manage a
resource, such as when and how it was created, file type and other
technical information, and who can access it.

We will cover metadata, especially while studying relational data.

    \subsubsection{Data Explosion}\label{data-explosion}

\begin{figure}
\centering
\includegraphics{attachment:image.png}
\caption{image.png}
\end{figure}

\begin{longtable}[]{@{}c@{}}
\toprule
\tabularnewline
\midrule
\endhead
\textbf{Data Explosion}\tabularnewline
\bottomrule
\end{longtable}

\begin{itemize}
\item
  The world's total, yearly data creation will reach
  \href{https://www.forbes.com/sites/andrewcave/2017/04/13/what-will-we-do-when-the-worlds-data-hits-163-zettabytes-in-2025/\#43b56b37349a}{163
  zettabytes in 2025.}
\item
  \textbf{DANGER:} Math before caffeine.
\item
  Zettabyte is \(1000^7 = 10^{21}.\)
\item
  World's population is/will be approximately \(10 * 10^{9} = 10^{10}\)
\item
  Data per person in 2025 =

  \begin{equation}
  \frac{1.63 * 10^2 * 10^{21}}{10^{10}} = \frac{1.63 * 10^{23}}{10^{10}} = 1.63 * 10^{13} = 16.3*10^12 = 16GB \ per \ person \ per \ year.
  \end{equation}
\item
  In 2025, "an average connected person anywhere in the world will
  interact with connected devices nearly 4,800 times per day -- one
  interaction every 18 seconds."
  \href{https://www.forbes.com/sites/andrewcave/2017/04/13/what-will-we-do-when-the-worlds-data-hits-163-zettabytes-in-2025/\#43b56b37349a}{Forbes}.
  Each interaction creates new data.
\end{itemize}

    \subsubsection{The Four Vs}\label{the-four-vs}

\begin{longtable}[]{@{}c@{}}
\toprule
\tabularnewline
\midrule
\endhead
\textbf{Four Vs of Data}\tabularnewline
\bottomrule
\end{longtable}

    Some driving factors: - Volume: - File/instance size: videos, images,
VR, ... - Sources: Internet-of-Things, events and monitoring, ...

\begin{itemize}
\tightlist
\item
  Velocity:

  \begin{itemize}
  \tightlist
  \item
    Monitoring and events from everything.
  \item
    Video and audio from everywhere.
  \end{itemize}
\item
  Veracity:

  \begin{itemize}
  \tightlist
  \item
    Measurement error and approximation.
  \item
    Incorrect configuration, entry, processing applications.
  \end{itemize}
\item
  Variety:

  \begin{itemize}
  \tightlist
  \item
    Base types: several base models for structured data.
  \item
    Every device emits its own event format.
  \item
    Multiple encodings and formats of text (mail, web page, tweet)
    audio, video, ...
  \end{itemize}
\end{itemize}

    \subsubsection{Or, Maybe Five Vs}\label{or-maybe-five-vs}

\begin{longtable}[]{@{}c@{}}
\toprule
\tabularnewline
\midrule
\endhead
Five Vs\tabularnewline
\bottomrule
\end{longtable}

    \begin{itemize}
\tightlist
\item
  Value: "It's like trying to find a needle in a haystack size pile of
  needles."
\end{itemize}

    \subsection{Big data landscape}\label{big-data-landscape}

\begin{longtable}[]{@{}c@{}}
\toprule
\tabularnewline
\midrule
\endhead
\textbf{Big Data Landscape}\tabularnewline
\bottomrule
\end{longtable}

    \subsection{Databases}\label{databases}

\subsubsection{Definitions}\label{definitions}

``A~database~is an organized collection of~data.~It is the collection
of~schemas,~tables,~queries, reports,~views, and other objects. The data
are typically organized to model aspects of reality in a way that
supports~processes~requiring information, such as modelling the
availability of rooms in hotels in a way that supports finding a hotel
with vacancies. (https://en.wikipedia.org/wiki/Database)''

``Systematically organized or structured repository of indexed
information (usually as a group of linked data files) that allows easy
retrieval, updating, analysis, and output of data. Stored usually in a
computer, this data could be in the form of graphics, reports, scripts,
tables, text, etc., representing almost every kind of information. Most
computer applications (including antivirus software, spreadsheets,
word-processors) are databases at their core.''
(http://www.businessdictionary.com/definition/database.html)

\subsubsection{Simplistic example}\label{simplistic-example}

\begin{longtable}[]{@{}c@{}}
\toprule
\tabularnewline
\midrule
\endhead
\textbf{Simplistic Example (and First Datamodel)}\tabularnewline
\bottomrule
\end{longtable}

\begin{itemize}
\tightlist
\item
  Entity -- Data about a ``Thing,'' e.g.

  \begin{itemize}
  \tightlist
  \item
    Person
  \item
    Web click
  \item
    Product
  \end{itemize}
\item
  Fields (Properties) -- The data describing, defining an entity. Often
  named and typed, e.g.

  \begin{itemize}
  \tightlist
  \item
    (Height, Integer)
  \item
    (Last name, String)
  \end{itemize}
\item
  Entity Set/Collection/Table

  \begin{itemize}
  \tightlist
  \item
    A group of things.
  \item
    Usually the same ''kind of entity''
  \end{itemize}
\item
  Relationships/Associations -- Links between entities, which convey
  semantic information, e.g.

  \begin{itemize}
  \tightlist
  \item
    Don IsA Professor
  \item
    Don Teaches \{COMS4111, COMSE6998)
  \end{itemize}
\end{itemize}

\subsubsection{Slightly Less Simplistic
Example}\label{slightly-less-simplistic-example}

\begin{longtable}[]{@{}c@{}}
\toprule
\tabularnewline
\midrule
\endhead
Slightly Less Simplistic Schema\_\_\tabularnewline
\bottomrule
\end{longtable}

\begin{longtable}[]{@{}c@{}}
\toprule
\tabularnewline
\midrule
\endhead
\textbf{Slightly, Slightly Less Simplistic Schema}\tabularnewline
\bottomrule
\end{longtable}

\begin{itemize}
\tightlist
\item
  All joking aside. A single database/datamodel can have

  \begin{itemize}
  \tightlist
  \item
    100s of entities.
  \item
    100s of relationships
  \item
    1000s of attributes.
  \end{itemize}
\end{itemize}

    \subsection{Database Management
Systems}\label{database-management-systems}

\begin{longtable}[]{@{}c@{}}
\toprule
\tabularnewline
\midrule
\endhead
\textbf{DBMS Taxonomy}\tabularnewline
\bottomrule
\end{longtable}

\begin{longtable}[]{@{}c@{}}
\toprule
\tabularnewline
\midrule
\endhead
\textbf{DBMS Engine Overview}\tabularnewline
\bottomrule
\end{longtable}

    \begin{itemize}
\item
  We will cover DBMS implementation in Module II.
\item
  To understand DBMS models and usage, we will primarily focus on 4
  database management systems:

  \begin{enumerate}
  \def\labelenumi{\arabic{enumi}.}
  \item
    Relational (MySQL)
  \item
    Graph Database (Neo4J)
  \item
    Document (DynamoDB, Firebase)
  \item
    Key/Value- data structure store (Redis)
  \end{enumerate}
\item
  Motivation:

  \begin{itemize}
  \tightlist
  \item
    https://db-engines.com/en/ranking adoption and growth.
  \item
    Diversity of base models covers core patterns in
    \emph{functionality} and \emph{performance.}
  \end{itemize}
\end{itemize}

\begin{longtable}[]{@{}c@{}}
\toprule
\tabularnewline
\midrule
\endhead
\href{https://db-engines.com/en/ranking_categories}{DBMS popularity
broken down by database model}\tabularnewline
\bottomrule
\end{longtable}

\begin{longtable}[]{@{}c@{}}
\toprule
\tabularnewline
\midrule
\endhead
\href{https://db-engines.com/en/ranking_categories}{Popularity changes
per category, August 2018}\tabularnewline
\bottomrule
\end{longtable}

    \begin{itemize}
\tightlist
\item
  Why start with MySQL? Popular, open source and free.
\end{itemize}

\begin{longtable}[]{@{}c@{}}
\toprule
\tabularnewline
\midrule
\endhead
\href{https://db-engines.com/en/ranking/relational+dbms}{DB-Engines
Ranking of Relational DBMS}\tabularnewline
\bottomrule
\end{longtable}

    \subsection{Data modeling}\label{data-modeling}

\subsubsection{Overview}\label{overview}

\begin{longtable}[]{@{}c@{}}
\toprule
\tabularnewline
\midrule
\endhead
\href{https://en.wikipedia.org/wiki/Data_model}{Data
Model}\tabularnewline
\bottomrule
\end{longtable}

\begin{itemize}
\item
  ``Data modeling (data modelling) is the analysis of data objects and
  their relationships to other data objects. Data modeling is often the
  first step in database design and object-oriented programming as the
  designers first create a conceptual model of how data items relate to
  each other. Data modeling involves a progression from conceptual model
  to logical model to physical schema.''
  (http://www.webopedia.com/TERM/D/data\_modeling.html)
\item
  Conceptual-Logical-Physical
\end{itemize}

\begin{longtable}[]{@{}c@{}}
\toprule
\tabularnewline
\midrule
\endhead
\href{https://en.wikipedia.org/wiki/Data_model}{Data
Model}\tabularnewline
\bottomrule
\end{longtable}

\begin{itemize}
\tightlist
\item
  ``What is a datamodel? A data model is a \textbf{notation} for
  describing data or information. The description generally consists of
  three parts:

  \begin{itemize}
  \tightlist
  \item
    Structure of data.
  \item
    Operations on the data.
  \item
    Constraints on the data.'' (Database Systems: The Complete Book (2nd
    Edition) by~Hector Garcia-Molina~(Author),~Jeffrey D.
    Ullman~(Author),~Jennifer Widom~(Author))
  \end{itemize}
\item
  \textbf{Notation:}

  \begin{itemize}
  \tightlist
  \item
    "A visual notation is a graphical representation. It consists of
    graphical symbols, their definitions, and a visual grammar. Some
    examples of graphical symbols are: lines, surfaces, volumes, textual
    labels and spatial relationships. These elements are used to build
    the visual vocabulary of a notation; Mind Maps for example, consist
    of lines and labels. Visual representations are effective because
    they convey information more concisely and precisely than language.
    They are also better remembered."
    (https://www.sciencedaily.com/releases/2013/07/130718161429.htm)
  \item
    Think "well-defined clipart" with grammar (rules, meaning)
  \end{itemize}
\end{itemize}

    \begin{longtable}[]{@{}c@{}}
\toprule
\tabularnewline
\midrule
\endhead
\textbf{Visual Notation}\tabularnewline
\bottomrule
\end{longtable}

\begin{longtable}[]{@{}c@{}}
\toprule
\tabularnewline
\midrule
\endhead
\textbf{Visual Notation}\tabularnewline
\bottomrule
\end{longtable}

\begin{itemize}
\tightlist
\item
  We will use
  \href{https://en.wikipedia.org/wiki/Entity\%E2\%80\%93relationship_model\#Crow's_foot_notation}{Crow's
  Foot Notation}

  \begin{itemize}
  \tightlist
  \item
    Simple, clear, etc.
  \item
    There are more powerful and complex notations, but getting carried
    away is easy.
  \item
    The textbook uses
    \href{https://en.wikipedia.org/wiki/Entity\%E2\%80\%93relationship_model\#Entity\%E2\%80\%93relationship_modeling}{Chen's
    Notation.}
  \end{itemize}
\end{itemize}

    \subsubsection{Modling occurs in many application
domains}\label{modling-occurs-in-many-application-domains}

Some randomly (by Google) selected examples.
\href{https://doi.org/10.1186/s40537-015-0024-1}{"Database application
model and its service for drug discovery in Model-driven architecture"}

 \href{http://geotech.com/envirodata}{Enviro Data}


\href{https://gds.blog.gov.uk/2013/10/31/government-as-a-data-model-what-i-learned-in-estonia/}{Government}
as a Data Model

    \subsection{Let's do this}\label{lets-do-this}

\subsubsection{Set up}\label{set-up}

\begin{itemize}
\item
  I downloaded
  \href{http://www.seanlahman.com/baseball-archive/statistics/}{Lahman's
  Baseball Database}
\item
  Imported in MySQL DB server on my laptop.
\item
  Full datamodel
\end{itemize}

\begin{longtable}[]{@{}c@{}}
\toprule
\tabularnewline
\midrule
\endhead
\textbf{Full Lahman Data Model}\tabularnewline
\bottomrule
\end{longtable}

\begin{itemize}
\tightlist
\item
  Interesting subset (for now and 1st project).
\end{itemize}

\begin{longtable}[]{@{}c@{}}
\toprule
\tabularnewline
\midrule
\endhead
\textbf{Initial Subset}\tabularnewline
\bottomrule
\end{longtable}

\begin{itemize}
\tightlist
\item
  Data model

  \begin{enumerate}
  \def\labelenumi{\arabic{enumi}.}
  \tightlist
  \item
    You can think of each one of the rectangles being a single comma
    separated value file.
  \item
    Entries in entity are column names and data types.
  \item
    Lines indicate how to use columns from one file to find columns in
    another file. For example

    \begin{enumerate}
    \def\labelenumii{\arabic{enumii}.}
    \tightlist
    \item
      playerID can be used to find name information in Master.
    \item
      playerID can be use to find batting information in Batting.
    \end{enumerate}
  \end{enumerate}
\item
  There are additional semantics that we will enforce (or the DBMS) will
  enforce.

  \begin{itemize}
  \tightlist
  \item
    Primary Key: A set of columns for which a given combination of
    values occurs at most once.
  \item
    Index: Helper data structure that speeds up operations.
  \item
    Integrity constraints: A record can exist in entity A only if there
    is a corresponding entry in entity B.
  \end{itemize}
\end{itemize}

\begin{longtable}[]{@{}c@{}}
\toprule
\tabularnewline
\midrule
\endhead
\textbf{Additional Semantics}\tabularnewline
\bottomrule
\end{longtable}

\begin{itemize}
\tightlist
\item
  Some queries:

  \begin{itemize}
  \tightlist
  \item
    Find a player by the unique playerID.
  \item
    Find all players that match a template.
  \item
    Find all of a player's batting information based on playerID.
  \item
    Given a playerID, return

    \begin{itemize}
    \tightlist
    \item
      nameLast, nameFirst, Throws, Bats, birthYear
    \item
      Career home runs, hits and batting average.
    \end{itemize}
  \item
    I want to know the 10 greatest career hitters whose last appearance
    was 1960 or later. A simple algorithm is

    \begin{enumerate}
    \def\labelenumi{\arabic{enumi}.}
    \tightlist
    \item
      For each unique playerID in Batting

      \begin{enumerate}
      \def\labelenumii{\arabic{enumii}.}
      \tightlist
      \item
        Find the batting records for each year.
      \item
        If one of the years \textgreater{}= 1960. Compute over all years

        \begin{enumerate}
        \def\labelenumiii{\arabic{enumiii}.}
        \tightlist
        \item
          Total at bats
        \item
          Toal hits
        \item
          Batting average
        \end{enumerate}
      \end{enumerate}
    \item
      For every result from above, use the playerID to look the name up
      in Master. Add to result.
    \item
      Sort result by batting average descending.
    \end{enumerate}
  \end{itemize}
\end{itemize}

    \subsubsection{Query I: Find Player by
PlayerID}\label{query-i-find-player-by-playerid}

\paragraph{CSV File Implementation}\label{csv-file-implementation}

    \begin{Verbatim}[commandchars=\\\{\}]
{\color{incolor}In [{\color{incolor}11}]:} \PY{k+kn}{import} \PY{n+nn}{csv}
         \PY{k+kn}{import} \PY{n+nn}{json}
         
         \PY{n}{data\PYZus{}dir} \PY{o}{=} \PY{l+s+s2}{\PYZdq{}}\PY{l+s+s2}{/Users/donaldferguson/Dropbox/ColumbiaCourse/Courses/Fall2018/W4111/Data/}\PY{l+s+s2}{\PYZdq{}}
         
         \PY{k}{def} \PY{n+nf}{findByPlayerID}\PY{p}{(}\PY{n+nb}{id}\PY{p}{)}\PY{p}{:}
             \PY{k}{with} \PY{n+nb}{open}\PY{p}{(}\PY{n}{data\PYZus{}dir} \PY{o}{+} \PY{l+s+s2}{\PYZdq{}}\PY{l+s+s2}{People.csv}\PY{l+s+s2}{\PYZdq{}}\PY{p}{)} \PY{k}{as} \PY{n}{csvfile}\PY{p}{:}
                 \PY{n}{reader} \PY{o}{=} \PY{n}{csv}\PY{o}{.}\PY{n}{DictReader}\PY{p}{(}\PY{n}{csvfile}\PY{p}{)}
                 \PY{k}{for} \PY{n}{row} \PY{o+ow}{in} \PY{n}{reader}\PY{p}{:}
                     \PY{k}{if} \PY{n}{row}\PY{p}{[}\PY{l+s+s1}{\PYZsq{}}\PY{l+s+s1}{playerID}\PY{l+s+s1}{\PYZsq{}}\PY{p}{]} \PY{o}{==} \PY{n+nb}{id}\PY{p}{:}
                         \PY{n}{result} \PY{o}{=} \PY{n}{row}
                         \PY{k}{return} \PY{n}{result}
                 \PY{k}{else}\PY{p}{:}
                     \PY{k}{return} \PY{k+kc}{None}
\end{Verbatim}


    \begin{Verbatim}[commandchars=\\\{\}]
{\color{incolor}In [{\color{incolor}2}]:} \PY{n}{playerID} \PY{o}{=} \PY{n+nb}{input}\PY{p}{(}\PY{l+s+s2}{\PYZdq{}}\PY{l+s+s2}{Enter a player}\PY{l+s+s2}{\PYZsq{}}\PY{l+s+s2}{s ID: }\PY{l+s+s2}{\PYZdq{}}\PY{p}{)}
        \PY{n+nb}{print}\PY{p}{(}\PY{l+s+s2}{\PYZdq{}}\PY{l+s+s2}{The player is }\PY{l+s+se}{\PYZbs{}n}\PY{l+s+s2}{\PYZdq{}}\PY{p}{,} \PY{n}{json}\PY{o}{.}\PY{n}{dumps}\PY{p}{(}\PY{n}{findByPlayerID}\PY{p}{(}\PY{n}{playerID}\PY{p}{)}\PY{p}{,} \PY{n}{indent}\PY{o}{=}\PY{l+m+mi}{3}\PY{p}{)}\PY{p}{)}
\end{Verbatim}


    \begin{Verbatim}[commandchars=\\\{\}]
Enter a player's ID: willite01
The player is 
 \{
   "playerID": "willite01",
   "birthYear": "1918",
   "birthMonth": "8",
   "birthDay": "30",
   "birthCountry": "USA",
   "birthState": "CA",
   "birthCity": "San Diego",
   "deathYear": "2002",
   "deathMonth": "7",
   "deathDay": "5",
   "deathCountry": "USA",
   "deathState": "FL",
   "deathCity": "Inverness",
   "nameFirst": "Ted",
   "nameLast": "Williams",
   "nameGiven": "Theodore Samuel",
   "weight": "205",
   "height": "75",
   "bats": "L",
   "throws": "R",
   "debut": "1939-04-20",
   "finalGame": "1960-09-28",
   "retroID": "willt103",
   "bbrefID": "willite01"
\}

    \end{Verbatim}

    \paragraph{Relational Database (MySQL)
Implementation}\label{relational-database-mysql-implementation}

    \begin{Verbatim}[commandchars=\\\{\}]
{\color{incolor}In [{\color{incolor}1}]:} \PY{k+kn}{import} \PY{n+nn}{pymysql}\PY{n+nn}{.}\PY{n+nn}{cursors}
        \PY{k+kn}{import} \PY{n+nn}{pandas} \PY{k}{as} \PY{n+nn}{pd}
        \PY{k+kn}{import} \PY{n+nn}{json}
        
        \PY{c+c1}{\PYZsh{} The database server is running somewhere in the network.}
        \PY{c+c1}{\PYZsh{} I must specify the IP address (HW server) and port number}
        \PY{c+c1}{\PYZsh{} (connection that SW server is listening on)}
        \PY{c+c1}{\PYZsh{} Also, I do not want to allow anyone to access the database}
        \PY{c+c1}{\PYZsh{} and different people have different permissions. So, the}
        \PY{c+c1}{\PYZsh{} client must log on.}
        
        
        \PY{c+c1}{\PYZsh{} Connect to the database over the network. Use the connection}
        \PY{c+c1}{\PYZsh{} to send commands to the DB.}
        \PY{n}{cnx} \PY{o}{=} \PY{n}{pymysql}\PY{o}{.}\PY{n}{connect}\PY{p}{(}\PY{n}{host}\PY{o}{=}\PY{l+s+s1}{\PYZsq{}}\PY{l+s+s1}{localhost}\PY{l+s+s1}{\PYZsq{}}\PY{p}{,}
                                     \PY{n}{user}\PY{o}{=}\PY{l+s+s1}{\PYZsq{}}\PY{l+s+s1}{dbuser}\PY{l+s+s1}{\PYZsq{}}\PY{p}{,}
                                     \PY{n}{password}\PY{o}{=}\PY{l+s+s1}{\PYZsq{}}\PY{l+s+s1}{dbuser}\PY{l+s+s1}{\PYZsq{}}\PY{p}{,}
                                     \PY{n}{db}\PY{o}{=}\PY{l+s+s1}{\PYZsq{}}\PY{l+s+s1}{lahman2017}\PY{l+s+s1}{\PYZsq{}}\PY{p}{,}
                                     \PY{n}{charset}\PY{o}{=}\PY{l+s+s1}{\PYZsq{}}\PY{l+s+s1}{utf8mb4}\PY{l+s+s1}{\PYZsq{}}\PY{p}{,}
                                     \PY{n}{cursorclass}\PY{o}{=}\PY{n}{pymysql}\PY{o}{.}\PY{n}{cursors}\PY{o}{.}\PY{n}{DictCursor}\PY{p}{)}
        
        \PY{k}{def} \PY{n+nf}{DBfindByPlayerID}\PY{p}{(}\PY{n+nb}{id}\PY{p}{)}\PY{p}{:}
            \PY{n}{cursor}\PY{o}{=}\PY{n}{cnx}\PY{o}{.}\PY{n}{cursor}\PY{p}{(}\PY{p}{)}
            \PY{n}{q} \PY{o}{=} \PY{l+s+s2}{\PYZdq{}}\PY{l+s+s2}{SELECT * FROM PEOPLE WHERE playerID=}\PY{l+s+s2}{\PYZsq{}}\PY{l+s+s2}{\PYZdq{}} \PY{o}{+} \PY{n+nb}{id} \PY{o}{+} \PY{l+s+s2}{\PYZdq{}}\PY{l+s+s2}{\PYZsq{}}\PY{l+s+s2}{;}\PY{l+s+s2}{\PYZdq{}}
            \PY{n+nb}{print} \PY{p}{(}\PY{l+s+s2}{\PYZdq{}}\PY{l+s+s2}{Query = }\PY{l+s+s2}{\PYZdq{}}\PY{p}{,} \PY{n}{q}\PY{p}{)}
            \PY{n}{cursor}\PY{o}{.}\PY{n}{execute}\PY{p}{(}\PY{n}{q}\PY{p}{)}\PY{p}{;}
            \PY{n}{r} \PY{o}{=} \PY{n}{cursor}\PY{o}{.}\PY{n}{fetchone}\PY{p}{(}\PY{p}{)}
            \PY{c+c1}{\PYZsh{}print(\PYZdq{}Query result = \PYZdq{}, r)}
            \PY{k}{return} \PY{n}{r}
\end{Verbatim}


    \begin{Verbatim}[commandchars=\\\{\}]
{\color{incolor}In [{\color{incolor} }]:} \PY{n}{playerID} \PY{o}{=} \PY{n+nb}{input}\PY{p}{(}\PY{l+s+s2}{\PYZdq{}}\PY{l+s+s2}{Enter a player}\PY{l+s+s2}{\PYZsq{}}\PY{l+s+s2}{s ID: }\PY{l+s+s2}{\PYZdq{}}\PY{p}{)}
        \PY{n+nb}{print}\PY{p}{(}\PY{l+s+s2}{\PYZdq{}}\PY{l+s+s2}{The player is }\PY{l+s+se}{\PYZbs{}n}\PY{l+s+s2}{\PYZdq{}}\PY{p}{,} \PY{n}{json}\PY{o}{.}\PY{n}{dumps}\PY{p}{(}\PY{n}{DBfindByPlayerID}\PY{p}{(}\PY{n}{playerID}\PY{p}{)}\PY{p}{,} \PY{n}{indent}\PY{o}{=}\PY{l+m+mi}{3}\PY{p}{)}\PY{p}{)}
\end{Verbatim}


    \paragraph{Comments}\label{comments}

\begin{enumerate}
\def\labelenumi{\arabic{enumi}.}
\tightlist
\item
  You have to specify an "open" or "connect" to the thing holding the
  data. 
\item
  CSV and Python is \textbf{imperative} or \textbf{control flow.} You
  have to explicitly code what you want to happen. 
\item
  The RDB is \textbf{declarative.}

  \begin{itemize}
  \tightlist
  \item
    You provide a statement that expresses what you want to happen.
  \item
    The database interprets and executes. 
  \end{itemize}
\item
  Performance:

  \begin{itemize}
  \tightlist
  \item
    The Python code loads and examines \texttt{O(n)} records, which is
    \texttt{19,371} in this example.
  \item
    The RDB loads and examines \texttt{O(log(n))} records, which is
    \texttt{15} in this example because of the \textbf{index.}
  \end{itemize}
\end{enumerate}

    \subsubsection{Query II: Find All Players Matching a
Template}\label{query-ii-find-all-players-matching-a-template}

\paragraph{RDB}\label{rdb}

    \begin{Verbatim}[commandchars=\\\{\}]
{\color{incolor}In [{\color{incolor}13}]:} \PY{k}{def} \PY{n+nf}{templateToWhereClause}\PY{p}{(}\PY{n}{t}\PY{p}{)}\PY{p}{:}
             \PY{n}{s} \PY{o}{=} \PY{l+s+s2}{\PYZdq{}}\PY{l+s+s2}{\PYZdq{}}
             \PY{k}{for} \PY{p}{(}\PY{n}{k}\PY{p}{,}\PY{n}{v}\PY{p}{)} \PY{o+ow}{in} \PY{n}{t}\PY{o}{.}\PY{n}{items}\PY{p}{(}\PY{p}{)}\PY{p}{:}
                 \PY{k}{if} \PY{n}{s} \PY{o}{!=} \PY{l+s+s2}{\PYZdq{}}\PY{l+s+s2}{\PYZdq{}}\PY{p}{:}
                     \PY{n}{s} \PY{o}{+}\PY{o}{=} \PY{l+s+s2}{\PYZdq{}}\PY{l+s+s2}{ AND }\PY{l+s+s2}{\PYZdq{}}
                 \PY{n}{s} \PY{o}{+}\PY{o}{=} \PY{n}{k} \PY{o}{+} \PY{l+s+s2}{\PYZdq{}}\PY{l+s+s2}{=}\PY{l+s+s2}{\PYZsq{}}\PY{l+s+s2}{\PYZdq{}} \PY{o}{+} \PY{n}{v} \PY{o}{+} \PY{l+s+s2}{\PYZdq{}}\PY{l+s+s2}{\PYZsq{}}\PY{l+s+s2}{\PYZdq{}}
         
             \PY{k}{if} \PY{n}{s} \PY{o}{!=} \PY{l+s+s2}{\PYZdq{}}\PY{l+s+s2}{\PYZdq{}}\PY{p}{:}
                 \PY{n}{s} \PY{o}{=} \PY{l+s+s2}{\PYZdq{}}\PY{l+s+s2}{WHERE }\PY{l+s+s2}{\PYZdq{}} \PY{o}{+} \PY{n}{s}\PY{p}{;}
         
             \PY{k}{return} \PY{n}{s}
         
         \PY{k}{def} \PY{n+nf}{DBfindByTemplate}\PY{p}{(}\PY{n}{t}\PY{p}{)}\PY{p}{:}
             \PY{n}{w} \PY{o}{=} \PY{n}{templateToWhereClause}\PY{p}{(}\PY{n}{t}\PY{p}{)}
             \PY{n}{cursor}\PY{o}{=}\PY{n}{cnx}\PY{o}{.}\PY{n}{cursor}\PY{p}{(}\PY{p}{)}
             \PY{n}{q} \PY{o}{=} \PY{l+s+s2}{\PYZdq{}}\PY{l+s+s2}{SELECT * FROM People }\PY{l+s+s2}{\PYZdq{}} \PY{o}{+} \PY{n}{w} \PY{o}{+} \PY{l+s+s2}{\PYZdq{}}\PY{l+s+s2}{;}\PY{l+s+s2}{\PYZdq{}}
             \PY{n+nb}{print} \PY{p}{(}\PY{l+s+s2}{\PYZdq{}}\PY{l+s+s2}{Query = }\PY{l+s+s2}{\PYZdq{}}\PY{p}{,} \PY{n}{q}\PY{p}{)}
             \PY{n}{cursor}\PY{o}{.}\PY{n}{execute}\PY{p}{(}\PY{n}{q}\PY{p}{)}\PY{p}{;}
             \PY{n}{r} \PY{o}{=} \PY{n}{cursor}\PY{o}{.}\PY{n}{fetchall}\PY{p}{(}\PY{p}{)}
             \PY{c+c1}{\PYZsh{}print(\PYZdq{}Query result = \PYZdq{}, r)}
             \PY{k}{return} \PY{n}{r}
\end{Verbatim}


    \begin{Verbatim}[commandchars=\\\{\}]
{\color{incolor}In [{\color{incolor}14}]:} \PY{c+c1}{\PYZsh{} \PYZob{} \PYZdq{}nameLast\PYZdq{} : \PYZdq{}Williams\PYZdq{}, \PYZdq{}nameFirst\PYZdq{} : \PYZdq{}Ted\PYZdq{} \PYZcb{}}
         \PY{n}{t} \PY{o}{=} \PY{n+nb}{input}\PY{p}{(}\PY{l+s+s2}{\PYZdq{}}\PY{l+s+s2}{Input a template: }\PY{l+s+s2}{\PYZdq{}}\PY{p}{)}
         \PY{n}{t} \PY{o}{=} \PY{n}{json}\PY{o}{.}\PY{n}{loads}\PY{p}{(}\PY{n}{t}\PY{p}{)}
         \PY{n}{DBfindByTemplate}\PY{p}{(}\PY{n}{t}\PY{p}{)}
\end{Verbatim}


    \begin{Verbatim}[commandchars=\\\{\}]
Input a template: \{ "nameLast" : "Williams", "nameFirst" : "Ted" \}
Query =  SELECT * FROM People WHERE nameLast='Williams' AND nameFirst='Ted';

    \end{Verbatim}

\begin{Verbatim}[commandchars=\\\{\}]
{\color{outcolor}Out[{\color{outcolor}14}]:} [\{'bats': 'L',
           'bbrefID': 'willite01',
           'birthCity': 'San Diego',
           'birthCountry': 'USA',
           'birthDay': '30',
           'birthMonth': '8',
           'birthState': 'CA',
           'birthYear': '1918',
           'deathCity': 'Inverness',
           'deathCountry': 'USA',
           'deathDay': '5',
           'deathMonth': '7',
           'deathState': 'FL',
           'deathYear': '2002',
           'debut': '1939-04-20',
           'finalGame': '1960-09-28',
           'height': 75,
           'nameFirst': 'Ted',
           'nameGiven': 'Theodore Samuel',
           'nameLast': 'Williams',
           'playerID': 'willite01',
           'retroID': 'willt103',
           'throws': 'R',
           'weight': '205'\}]
\end{Verbatim}
            
    \paragraph{CSV}\label{csv}

    \begin{Verbatim}[commandchars=\\\{\}]
{\color{incolor}In [{\color{incolor} }]:} \PY{c+c1}{\PYZsh{} Your code goes here for homework 1.}
        \PY{c+c1}{\PYZsh{} Template is of the form \PYZob{} \PYZdq{}nameLast\PYZdq{} : \PYZdq{}Williams\PYZdq{}, \PYZdq{}nameFirst\PYZdq{} : \PYZdq{}Ted\PYZdq{} \PYZcb{}}
        \PY{c+c1}{\PYZsh{} You support arbitrary templates of the form \PYZob{} \PYZdq{}col1\PYZdq{} : \PYZdq{}value1\PYZdq{}, ... \PYZcb{}}
        \PY{c+c1}{\PYZsh{} Return all entries that match the template.}
\end{Verbatim}


    \paragraph{Comments}\label{comments}

\begin{enumerate}
\def\labelenumi{\arabic{enumi}.}
\tightlist
\item
  Most of the complexity in my Python code is converting from a Python
  dictionary to SQL WHERE clause. 
\item
  I did this the hard way. There is a simpler, more usable approach.
\end{enumerate}

    \begin{Verbatim}[commandchars=\\\{\}]
{\color{incolor}In [{\color{incolor}15}]:} \PY{c+c1}{\PYZsh{} This function runs any query.}
         \PY{k}{def} \PY{n+nf}{run\PYZus{}q}\PY{p}{(}\PY{n}{q}\PY{p}{)}\PY{p}{:}
             \PY{n}{cursor}\PY{o}{=}\PY{n}{cnx}\PY{o}{.}\PY{n}{cursor}\PY{p}{(}\PY{p}{)}
             \PY{n+nb}{print} \PY{p}{(}\PY{l+s+s2}{\PYZdq{}}\PY{l+s+s2}{Query = }\PY{l+s+s2}{\PYZdq{}}\PY{p}{,} \PY{n}{q}\PY{p}{)}
             \PY{n}{cursor}\PY{o}{.}\PY{n}{execute}\PY{p}{(}\PY{n}{q}\PY{p}{)}\PY{p}{;}
             \PY{n}{r} \PY{o}{=} \PY{n}{cursor}\PY{o}{.}\PY{n}{fetchall}\PY{p}{(}\PY{p}{)}
             \PY{c+c1}{\PYZsh{}print(\PYZdq{}Query result = \PYZdq{}, r)}
             \PY{k}{return} \PY{n}{r}
\end{Verbatim}


    \begin{Verbatim}[commandchars=\\\{\}]
{\color{incolor}In [{\color{incolor}16}]:} \PY{n}{t} \PY{o}{=} \PY{p}{\PYZob{}} \PY{l+s+s2}{\PYZdq{}}\PY{l+s+s2}{nameLast}\PY{l+s+s2}{\PYZdq{}} \PY{p}{:} \PY{l+s+s2}{\PYZdq{}}\PY{l+s+s2}{Williams}\PY{l+s+s2}{\PYZdq{}}\PY{p}{,} \PY{l+s+s2}{\PYZdq{}}\PY{l+s+s2}{Throws}\PY{l+s+s2}{\PYZdq{}} \PY{p}{:} \PY{l+s+s2}{\PYZdq{}}\PY{l+s+s2}{L}\PY{l+s+s2}{\PYZdq{}} \PY{p}{\PYZcb{}}
         \PY{n}{q} \PY{o}{=} \PY{l+s+s2}{\PYZdq{}}\PY{l+s+s2}{SELECT * FROM People }\PY{l+s+s2}{\PYZdq{}} \PY{o}{+} \PY{n}{templateToWhereClause}\PY{p}{(}\PY{n}{t}\PY{p}{)}
         \PY{n}{run\PYZus{}q}\PY{p}{(}\PY{n}{q}\PY{p}{)}
\end{Verbatim}


    \begin{Verbatim}[commandchars=\\\{\}]
Query =  SELECT * FROM People WHERE nameLast='Williams' AND Throws='L'

    \end{Verbatim}

\begin{Verbatim}[commandchars=\\\{\}]
{\color{outcolor}Out[{\color{outcolor}16}]:} [\{'bats': 'R',
           'bbrefID': 'williac01',
           'birthCity': 'Montclair',
           'birthCountry': 'USA',
           'birthDay': '18',
           'birthMonth': '3',
           'birthState': 'NJ',
           'birthYear': '1917',
           'deathCity': 'Fort Myers',
           'deathCountry': 'USA',
           'deathDay': '16',
           'deathMonth': '9',
           'deathState': 'FL',
           'deathYear': '1999',
           'debut': '1940-07-15',
           'finalGame': '1946-04-22',
           'height': 74,
           'nameFirst': 'Ace',
           'nameGiven': 'Robert Fulton',
           'nameLast': 'Williams',
           'playerID': 'williac01',
           'retroID': 'willa103',
           'throws': 'L',
           'weight': '174'\},
          \{'bats': 'L',
           'bbrefID': 'willicy01',
           'birthCity': 'Wadena',
           'birthCountry': 'USA',
           'birthDay': '21',
           'birthMonth': '12',
           'birthState': 'IN',
           'birthYear': '1887',
           'deathCity': 'Eagle River',
           'deathCountry': 'USA',
           'deathDay': '23',
           'deathMonth': '4',
           'deathState': 'WI',
           'deathYear': '1974',
           'debut': '1912-07-18',
           'finalGame': '1930-09-22',
           'height': 74,
           'nameFirst': 'Cy',
           'nameGiven': 'Fred',
           'nameLast': 'Williams',
           'playerID': 'willicy01',
           'retroID': 'willc103',
           'throws': 'L',
           'weight': '180'\},
          \{'bats': 'L',
           'bbrefID': 'willida05',
           'birthCity': 'Brooklyn',
           'birthCountry': 'USA',
           'birthDay': '28',
           'birthMonth': '2',
           'birthState': 'NY',
           'birthYear': '1958',
           'deathCity': '',
           'deathCountry': '',
           'deathDay': '',
           'deathMonth': '',
           'deathState': '',
           'deathYear': '',
           'debut': '1981-09-19',
           'finalGame': '1983-10-02',
           'height': 71,
           'nameFirst': 'Dallas',
           'nameGiven': 'Dallas McKinley',
           'nameLast': 'Williams',
           'playerID': 'willida05',
           'retroID': 'willd101',
           'throws': 'L',
           'weight': '165'\},
          \{'bats': 'R',
           'bbrefID': 'willida02',
           'birthCity': 'Scranton',
           'birthCountry': 'USA',
           'birthDay': '7',
           'birthMonth': '2',
           'birthState': 'PA',
           'birthYear': '1881',
           'deathCity': 'Hot Springs',
           'deathCountry': 'USA',
           'deathDay': '25',
           'deathMonth': '4',
           'deathState': 'AR',
           'deathYear': '1918',
           'debut': '1902-07-02',
           'finalGame': '1902-08-03',
           'height': 71,
           'nameFirst': 'Dave',
           'nameGiven': 'David Owen',
           'nameLast': 'Williams',
           'playerID': 'willida02',
           'retroID': 'willd103',
           'throws': 'L',
           'weight': '167'\},
          \{'bats': 'L',
           'bbrefID': 'willida07',
           'birthCity': 'Anchorage',
           'birthCountry': 'USA',
           'birthDay': '12',
           'birthMonth': '3',
           'birthState': 'AK',
           'birthYear': '1979',
           'deathCity': '',
           'deathCountry': '',
           'deathDay': '',
           'deathMonth': '',
           'deathState': '',
           'deathYear': '',
           'debut': '2001-06-06',
           'finalGame': '2007-09-24',
           'height': 75,
           'nameFirst': 'David',
           'nameGiven': 'David Aaron',
           'nameLast': 'Williams',
           'playerID': 'willida07',
           'retroID': 'willd002',
           'throws': 'L',
           'weight': '215'\},
          \{'bats': 'L',
           'bbrefID': 'willigu02',
           'birthCity': 'Omaha',
           'birthCountry': 'USA',
           'birthDay': '7',
           'birthMonth': '5',
           'birthState': 'NE',
           'birthYear': '1888',
           'deathCity': 'Sterling',
           'deathCountry': 'USA',
           'deathDay': '16',
           'deathMonth': '4',
           'deathState': 'IL',
           'deathYear': '1964',
           'debut': '1911-04-12',
           'finalGame': '1915-06-18',
           'height': 72,
           'nameFirst': 'Gus',
           'nameGiven': 'August Joseph',
           'nameLast': 'Williams',
           'playerID': 'willigu02',
           'retroID': 'willg103',
           'throws': 'L',
           'weight': '185'\},
          \{'bats': 'R',
           'bbrefID': 'willije02',
           'birthCity': 'Canberra',
           'birthCountry': 'Australia',
           'birthDay': '6',
           'birthMonth': '6',
           'birthState': 'Capital Territory',
           'birthYear': '1972',
           'deathCity': '',
           'deathCountry': '',
           'deathDay': '',
           'deathMonth': '',
           'deathState': '',
           'deathYear': '',
           'debut': '1999-09-12',
           'finalGame': '2002-09-28',
           'height': 72,
           'nameFirst': 'Jeff',
           'nameGiven': 'Jeffrey F.',
           'nameLast': 'Williams',
           'playerID': 'willije01',
           'retroID': 'willj002',
           'throws': 'L',
           'weight': '185'\},
          \{'bats': 'R',
           'bbrefID': 'willile01',
           'birthCity': 'Aurora',
           'birthCountry': 'USA',
           'birthDay': '9',
           'birthMonth': '3',
           'birthState': 'MO',
           'birthYear': '1893',
           'deathCity': 'Laguna Beach',
           'deathCountry': 'USA',
           'deathDay': '4',
           'deathMonth': '11',
           'deathState': 'CA',
           'deathYear': '1959',
           'debut': '1913-09-17',
           'finalGame': '1920-09-25',
           'height': 69,
           'nameFirst': 'Lefty',
           'nameGiven': 'Claude Preston',
           'nameLast': 'Williams',
           'playerID': 'willile01',
           'retroID': 'willl104',
           'throws': 'L',
           'weight': '160'\},
          \{'bats': 'L',
           'bbrefID': 'willile03',
           'birthCity': 'Macon',
           'birthCountry': 'USA',
           'birthDay': '2',
           'birthMonth': '12',
           'birthState': 'GA',
           'birthYear': '1905',
           'deathCity': 'Atlanta',
           'deathCountry': 'USA',
           'deathDay': '20',
           'deathMonth': '11',
           'deathState': 'GA',
           'deathYear': '1984',
           'debut': '1926-05-26',
           'finalGame': '1926-09-06',
           'height': 70,
           'nameFirst': 'Leon',
           'nameGiven': 'Leon Theo',
           'nameLast': 'Williams',
           'playerID': 'willile03',
           'retroID': 'willl101',
           'throws': 'L',
           'weight': '154'\},
          \{'bats': 'L',
           'bbrefID': 'willima02',
           'birthCity': 'Elmira',
           'birthCountry': 'USA',
           'birthDay': '28',
           'birthMonth': '7',
           'birthState': 'NY',
           'birthYear': '1953',
           'deathCity': '',
           'deathCountry': '',
           'deathDay': '',
           'deathMonth': '',
           'deathState': '',
           'deathYear': '',
           'debut': '1977-05-20',
           'finalGame': '1977-05-22',
           'height': 72,
           'nameFirst': 'Mark',
           'nameGiven': 'Mark Westley',
           'nameLast': 'Williams',
           'playerID': 'willima02',
           'retroID': 'willm105',
           'throws': 'L',
           'weight': '180'\},
          \{'bats': 'B',
           'bbrefID': 'willima09',
           'birthCity': 'Virginia Beach',
           'birthCountry': 'USA',
           'birthDay': '12',
           'birthMonth': '4',
           'birthState': 'VA',
           'birthYear': '1971',
           'deathCity': '',
           'deathCountry': '',
           'deathDay': '',
           'deathMonth': '',
           'deathState': '',
           'deathYear': '',
           'debut': '2000-04-05',
           'finalGame': '2000-04-29',
           'height': 72,
           'nameFirst': 'Matt',
           'nameGiven': 'Matt Taylor',
           'nameLast': 'Williams',
           'playerID': 'willima06',
           'retroID': 'willm006',
           'throws': 'L',
           'weight': '185'\},
          \{'bats': 'L',
           'bbrefID': 'willimi02',
           'birthCity': 'Santa Ana',
           'birthCountry': 'USA',
           'birthDay': '17',
           'birthMonth': '11',
           'birthState': 'CA',
           'birthYear': '1964',
           'deathCity': '',
           'deathCountry': '',
           'deathDay': '',
           'deathMonth': '',
           'deathState': '',
           'deathYear': '',
           'debut': '1986-04-09',
           'finalGame': '1997-05-10',
           'height': 75,
           'nameFirst': 'Mitch',
           'nameGiven': 'Mitchell Steven',
           'nameLast': 'Williams',
           'playerID': 'willimi02',
           'retroID': 'willm002',
           'throws': 'L',
           'weight': '180'\},
          \{'bats': 'L',
           'bbrefID': 'willini01',
           'birthCity': 'Galveston',
           'birthCountry': 'USA',
           'birthDay': '8',
           'birthMonth': '9',
           'birthState': 'TX',
           'birthYear': '1993',
           'deathCity': '',
           'deathCountry': '',
           'deathDay': '',
           'deathMonth': '',
           'deathState': '',
           'deathYear': '',
           'debut': '2017-06-30',
           'finalGame': '2017-10-01',
           'height': 75,
           'nameFirst': 'Nick',
           'nameGiven': 'Billy Nicholas',
           'nameLast': 'Williams',
           'playerID': 'willini01',
           'retroID': 'willn001',
           'throws': 'L',
           'weight': '195'\},
          \{'bats': 'R',
           'bbrefID': 'willipo01',
           'birthCity': 'Bowdoinham',
           'birthCountry': 'USA',
           'birthDay': '19',
           'birthMonth': '5',
           'birthState': 'ME',
           'birthYear': '1874',
           'deathCity': 'Topsham',
           'deathCountry': 'USA',
           'deathDay': '4',
           'deathMonth': '8',
           'deathState': 'ME',
           'deathYear': '1959',
           'debut': '1898-09-14',
           'finalGame': '1903-09-15',
           'height': 71,
           'nameFirst': 'Pop',
           'nameGiven': 'Walter Merrill',
           'nameLast': 'Williams',
           'playerID': 'willipo01',
           'retroID': 'willp101',
           'throws': 'L',
           'weight': '190'\},
          \{'bats': 'L',
           'bbrefID': 'willira01',
           'birthCity': 'Harlingen',
           'birthCountry': 'USA',
           'birthDay': '18',
           'birthMonth': '9',
           'birthState': 'TX',
           'birthYear': '1975',
           'deathCity': '',
           'deathCountry': '',
           'deathDay': '',
           'deathMonth': '',
           'deathState': '',
           'deathYear': '',
           'debut': '2004-09-11',
           'finalGame': '2011-08-05',
           'height': 75,
           'nameFirst': 'Randy',
           'nameGiven': 'Randall Duane',
           'nameLast': 'Williams',
           'playerID': 'willira01',
           'retroID': 'willr003',
           'throws': 'L',
           'weight': '200'\}]
\end{Verbatim}
            
    \subsubsection{Query III: Find the 10 Best Hitters in Modern
Baseball}\label{query-iii-find-the-10-best-hitters-in-modern-baseball}

\paragraph{Definition/Algorithm}\label{definitionalgorithm}

I want to know the 10 greatest career hitters whose last appearance was
1960 or later. A simple algorithm is 1. For each unique playerID in
Batting 1. Find the batting records for each year. 1. If one of the
years \textgreater{}= 1960. Compute over all years 1. Total at bats 1.
Toal hits 1. Batting average (total hits/total at bats) 1. For every
result from above, use the playerID to look the name up in Master. Add
to result. 1. Sort result by batting average descending.

    \paragraph{Relational Database
Implementation}\label{relational-database-implementation}

    \begin{Verbatim}[commandchars=\\\{\}]
{\color{incolor}In [{\color{incolor}1}]:} \PY{n}{q} \PY{o}{=} \PY{l+s+s2}{\PYZdq{}}\PY{l+s+s2}{SELECT }\PY{l+s+se}{\PYZbs{}}
        \PY{l+s+s2}{        Batting.playerID, }\PY{l+s+se}{\PYZbs{}}
        \PY{l+s+s2}{        (SELECT People.nameFirst FROM People WHERE People.playerID=Batting.playerID) as first\PYZus{}name, }\PY{l+s+se}{\PYZbs{}}
        \PY{l+s+s2}{        (SELECT People.nameLast FROM People WHERE People.playerID=Batting.playerID) as last\PYZus{}name, }\PY{l+s+se}{\PYZbs{}}
        \PY{l+s+s2}{        sum(Batting.h)/sum(batting.ab) as career\PYZus{}average, }\PY{l+s+se}{\PYZbs{}}
        \PY{l+s+s2}{        sum(Batting.h) as career\PYZus{}hits, }\PY{l+s+se}{\PYZbs{}}
        \PY{l+s+s2}{        sum(Batting.ab) as career\PYZus{}at\PYZus{}bats,}\PY{l+s+se}{\PYZbs{}}
        \PY{l+s+s2}{        min(Batting.yearID) as first\PYZus{}year, }\PY{l+s+se}{\PYZbs{}}
        \PY{l+s+s2}{        max(Batting.yearID) as last\PYZus{}year }\PY{l+s+se}{\PYZbs{}}
        \PY{l+s+s2}{        FROM }\PY{l+s+se}{\PYZbs{}}
        \PY{l+s+s2}{        Batting }\PY{l+s+se}{\PYZbs{}}
        \PY{l+s+s2}{        GROUP BY }\PY{l+s+se}{\PYZbs{}}
        \PY{l+s+s2}{        playerId }\PY{l+s+se}{\PYZbs{}}
        \PY{l+s+s2}{        HAVING }\PY{l+s+se}{\PYZbs{}}
        \PY{l+s+s2}{        career\PYZus{}at\PYZus{}bats \PYZgt{} 200 AND last\PYZus{}year \PYZgt{}= 1960 }\PY{l+s+se}{\PYZbs{}}
        \PY{l+s+s2}{        ORDER BY }\PY{l+s+se}{\PYZbs{}}
        \PY{l+s+s2}{        career\PYZus{}average DESC }\PY{l+s+se}{\PYZbs{}}
        \PY{l+s+s2}{        LIMIT 10;}\PY{l+s+s2}{\PYZdq{}}
        
        \PY{n}{r} \PY{o}{=} \PY{n}{run\PYZus{}q}\PY{p}{(}\PY{n}{q}\PY{p}{)}
        \PY{n}{r} \PY{o}{=} \PY{n}{pd}\PY{o}{.}\PY{n}{DataFrame}\PY{o}{.}\PY{n}{from\PYZus{}dict}\PY{p}{(}\PY{n}{r}\PY{p}{)}
        \PY{n}{r}
\end{Verbatim}


    \begin{Verbatim}[commandchars=\\\{\}]

        ---------------------------------------------------------------------------

        NameError                                 Traceback (most recent call last)

        <ipython-input-1-9b7e5912c06f> in <module>()
          1 q = "SELECT         Batting.playerID,         (SELECT People.nameFirst FROM People WHERE People.playerID=Batting.playerID) as first\_name,         (SELECT People.nameLast FROM People WHERE People.playerID=Batting.playerID) as last\_name,         sum(Batting.h)/sum(batting.ab) as career\_average,         sum(Batting.h) as career\_hits,         sum(Batting.ab) as career\_at\_bats,        min(Batting.yearID) as first\_year,         max(Batting.yearID) as last\_year         FROM         Batting         GROUP BY         playerId         HAVING         career\_at\_bats > 200 AND last\_year >= 1960         ORDER BY         career\_average DESC         LIMIT 10;"
          2 
    ----> 3 r = run\_q(q)
          4 r = pd.DataFrame.from\_dict(r)
          5 r


        NameError: name 'run\_q' is not defined

    \end{Verbatim}

    \paragraph{CSV}\label{csv}

    \begin{Verbatim}[commandchars=\\\{\}]
{\color{incolor}In [{\color{incolor} }]:} \PY{c+c1}{\PYZsh{} You guessed it.}
        \PY{c+c1}{\PYZsh{} Write a Python program that}
        \PY{c+c1}{\PYZsh{} Reads People}
        \PY{c+c1}{\PYZsh{} Reads Batting}
        \PY{c+c1}{\PYZsh{} Computes the answer I produced using SQL.}
        \PY{c+c1}{\PYZsh{}}
        
        \PY{c+c1}{\PYZsh{} Your code goes here.}
\end{Verbatim}


    \paragraph{Comments}\label{comments}

\begin{itemize}
\item
  The book and the web have a lot of information comparing databases
  versus applications and files.

  \begin{itemize}
  \tightlist
  \item
    Ramakrishnan and Gehrke 1.3, 1.4
  \item
    \href{https://stackoverflow.com/questions/2356851/database-vs-flat-files}{From
    stack overflow}

    \begin{enumerate}
    \def\labelenumi{\arabic{enumi}.}
    \tightlist
    \item
      Databases can handle querying tasks, so you don't have to walk
      over files manually.
    \item
      Databases can handle very complicated queries.
    \item
      Databases can handle indexing tasks, so if tasks like get record
      with id = x can be VERY fast
    \item
      Databases can handle multiprocess/multithreaded access.
    \item
      Databases can handle access from network
    \item
      Databases can watch for data integrity
    \item
      Databases can update data easily (see 1) )
    \item
      Databases are reliable
    \item
      Databases can handle transactions and concurrent access
    \item
      Databases + ORMs let you manipulate data in very programmer
      friendly way.
    \end{enumerate}
  \end{itemize}
\item
  These explanations are vague and abstract until you have tried doing
  more complex things with CSV files.
\item
  What I am asking you to do is relatively simple, but is still tedious
  without using a DB engine.
\end{itemize}

    \subsection{The Applications We Will Build (Parts
Of)}\label{the-applications-we-will-build-parts-of}

    \subsubsection{Interactive Web
Application}\label{interactive-web-application}

\begin{itemize}
\item
  We have data on Major League Baseball teams and players.
\item
  We want to allow users to:

  \begin{itemize}
  \tightlist
  \item
    Find, query and display information.
  \item
    Update information.
  \end{itemize}
\end{itemize}

\begin{longtable}[]{@{}c@{}}
\toprule
\tabularnewline
\midrule
\endhead
\textbf{Example User Interface}\tabularnewline
\bottomrule
\end{longtable}

\begin{itemize}
\tightlist
\item
  The UI task is to design and implement a baseball dashboard with three
  UI controls:

  \begin{itemize}
  \tightlist
  \item
    Select the resource(s) to query or update (Players, Batting,
    Appearances)
  \item
    Find an instance of the resource if you know the identifying fields.
  \item
    Create an \emph{ad hoc} query to locate resource instances based on
    columns and values. Choose which columns you want in responses.
  \end{itemize}
\item
  From
  \href{https://www.techopedia.com/definition/30581/ad-hoc-query-sql-programming}{Techopedia}:
  "As the word 'ad hoc' suggests, this type of query is designed for a
  'particular purpose,' which is in contrast to a predefined query ...
  An ad hoc query does not reside in the system for a long time and is
  created dynamically on demand by the user."

  \begin{itemize}
  \tightlist
  \item
    You can write a program or function for a predefined query, e.g.
    \texttt{findByLastName(ln).} The set of functions you write
    implements the predefined queries.
  \item
    \emph{Ad hoc} allows the user to enter arbitrary queries you did no
    anticipate.
  \end{itemize}
\item
  We will primarily focus on the data aspects, because building the UI,
  business logic, HTML, etc. can be very, very complex. Implementing
  these applications is a one or two semester course itself.
\end{itemize}

\begin{longtable}[]{@{}c@{}}
\toprule
\tabularnewline
\midrule
\endhead
\textbf{Application Structure}\tabularnewline
\bottomrule
\end{longtable}

    \subsubsection{Simple Business Intelligence, Data Visualization and
Decision
Support}\label{simple-business-intelligence-data-visualization-and-decision-support}

\begin{itemize}
\item
  The core elements of the RDB approach are very good for interactive
  applications, ad hoc queries and ensuring "data integrity." "Data
  integrity is the maintenance of, and the assurance of the accuracy and
  consistency of, ... and is a critical aspect to the design,
  implementation and usage of any system which stores, processes, or
  retrieves data." (https://en.wikipedia.org/wiki/Data\_integrity)
\item
  We will see how data modeling, schema definition, constraints, etc.
  enable "data integrity."
\item
  The data models and data design for analysis and decision support are
  much, much different. We will see some examples, e.g.

  \begin{itemize}
  \tightlist
  \item
    OLAP Cube
  \item
    Pivot Table
  \item
    ...
  \end{itemize}
\end{itemize}

\begin{longtable}[]{@{}c@{}}
\toprule
\tabularnewline
\midrule
\endhead
\href{https://aws.amazon.com/quicksight/}{AWS Quicksight Charting and
Analysis}\tabularnewline
\bottomrule
\end{longtable}

    "In database engineering in computing, OLAP cube is a term that
typically refers to a multi-dimensional array of data.

OLAP is an acronym for online analytical processing, which is a
computer-based technique of analyzing data to look for insights. The
term cube here refers to a multi-dimensional dataset, which is also
sometimes called a hypercube if the number of dimensions is greater than
3." (https://en.wikipedia.org/wiki/OLAP\_cube)

\begin{longtable}[]{@{}c@{}}
\toprule
\tabularnewline
\midrule
\endhead
\textbf{OLAP Cubes}\tabularnewline
\bottomrule
\end{longtable}

    \subsubsection{\texorpdfstring{Machine Learning: Very Simple
\href{https://en.wikipedia.org/wiki/Moneyball}{Moneyball}}{Machine Learning: Very Simple Moneyball}}\label{machine-learning-very-simple-moneyball}

    \begin{longtable}[]{@{}c@{}}
\toprule
\tabularnewline
\midrule
\endhead
\textbf{Neurons and Perceptrons}\tabularnewline
\bottomrule
\end{longtable}

    We will 1. Shape and modify the baseball data to enable machine
learning. 1. Use \href{http://scikit-learn.org/stable/}{scikit-learn} to
try to predict wins versus loses as a function of performance statistics
1. Shape information about individual players to allows us to build
"fantasy teams" and see how they would do.

\begin{longtable}[]{@{}c@{}}
\toprule
\tabularnewline
\midrule
\endhead
\textbf{Moneyball}\tabularnewline
\bottomrule
\end{longtable}

    \subsubsection{Six Degrees of Kevin
Bacon}\label{six-degrees-of-kevin-bacon}

"Six Degrees of Kevin Bacon is a parlor game based on the "six degrees
of separation" concept, which posits that any two people on Earth are
six or fewer acquaintance links apart. Movie buffs challenge each other
to find the shortest path between an arbitrary actor and prolific actor
Kevin Bacon. It rests on the assumption that anyone involved in the
Hollywood film industry can be linked through their film roles to Bacon
within six steps."
(https://en.wikipedia.org/wiki/Six\_Degrees\_of\_Kevin\_Bacon)

\begin{longtable}[]{@{}c@{}}
\toprule
\tabularnewline
\midrule
\endhead
\href{http://www.markrobinsonwrites.com/the-music-that-makes-me-dance/2018/3/11/movie-morsel-six-degrees-of-kevin-bacon}{Six
Degrees of Kevin Bacon}\tabularnewline
\bottomrule
\end{longtable}

    \begin{itemize}
\tightlist
\item
  We will learn how to model and solve the problem using a prebuilt
  Neo4J Graph Database.
\end{itemize}

\begin{longtable}[]{@{}c@{}}
\toprule
\tabularnewline
\midrule
\endhead
\textbf{Six Degrees of Kevin Bacon}\tabularnewline
\bottomrule
\end{longtable}

\begin{itemize}
\tightlist
\item
  We will also

  \begin{itemize}
  \tightlist
  \item
    Upload baseball to build graph database
  \item
    Perform graph queries based on players and teams.
  \end{itemize}
\end{itemize}

\begin{longtable}[]{@{}c@{}}
\toprule
\tabularnewline
\midrule
\endhead
\textbf{Baseball Graph Database}\tabularnewline
\bottomrule
\end{longtable}

    \subsection{What is next}\label{what-is-next}

\begin{enumerate}
\def\labelenumi{\arabic{enumi}.}
\tightlist
\item
  Join collaboration environments (section 1.4) 
\item
  Set up the development environment you want to use (section 1.6) 
\end{enumerate}


    % Add a bibliography block to the postdoc
    
    
    
    \end{document}
